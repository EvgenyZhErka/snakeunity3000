\chapter{\label{ch:ch01}ОБЗОР ПРОГРАНЫХ СРЕДСТВ ДЛЯ РЕАЛИЗАЦИИ ИГРЫ <<SNAKE>>} % Нужно сделать главу в содержании заглавными буквами

\section{\label{sec:ch01/sec01}Общие сведения об игре "Snake"}

История игры «Змейка» началась за несколько лет до появления первых мобильных телефонов. В 1977 году компания Gremlin Industries выпустила игровой автомат Hustle, рассчитанный на одного или двух игроков, в которой нужно было управлять «змейками», направляя их на бессистемно появляющиеся цели. Для победы нужно было заполучить больше очков, чем у оппонента, преграждая по ходу игры ему путь к новым целям (в случае многопользовательской игры), или просто побить установленный на игровом автомате рекорд. В 1984 году Gremlin Industries была вынуждена закрыться, но игра Hustle начала набирать обороты: сначала появился порт для компьютеров TRS-80, затем для Commodore PET и Apple II.

Оригинальная «Змейка» (Snake) от Nokia появилась в 1997 году благодаря стараниями разработчика Танели Орманто. В том же году компания выпустила первый телефон с этой игрой — Nokia 6110. Уже тогда игра была многопользовательской: телефоны общались через ИК-порты, ведь ни Bluetooth, ни тем более Wi-Fi в телефонах в то время не было. Сама змейка состояла из чёрных квадратов и могла двигаться в четырёх направлениях. Игровая зона, по которой передвигалось пресмыкающееся, была ограничена размерами экрана телефона: при ударе головы змейки о край телефона игра завершалась. «Змейка» приобрела невероятную популярность, сравнимую разве что с популярностью современных хитов «Angry Birds» и «Cut the Rope».

\subsection{\label{subsec:ch01/sec01/sub01}Правила игры}

\begin{enumerate}
    \item Змейка начинает движение с небольшой длиной и появляется на игровом поле.
    \item Игроку необходимо управлять движением змейки с помощью клавиатуры или других устройств управления.
    \item Змейка движется по игровому полю и не может остановиться, пока не столкнется с препятствием или не упрется в себя.
    \item Цель игры - съесть как можно больше блоков еды, которые появляются на игровом поле.
    \item По мере поедания еды, змейка увеличивается в длину и становится более длинной и сложной для управления.
    \item Игра заканчивается, если змейка сталкивается с самой собой или со стенами игрового поля.
\end{enumerate}

\section{\label{subsec:ch01/sec01/sub02}Основные характеристики игрового движка Unity}
Unity — это кроссплатформенный игровой движок, созданный компанией Unity Technologies в 2005 году. С его помощью делают одиночные и многопользовательские игры с современной 2D- и 3D-графикой для таких платформ, как ПК, PlayStation, Xbox, Nintendo Switch.


\textbf{Преимущества:}

\begin{enumerate}
    \item Широкие возможности. Unity предоставляет разработчикам множество функций и инструментов для создания игр. Он имеет готовые ресурсы, такие как сцены, префабы, анимации, которые можно использовать для создания игрового мира.
    \item Кроссплатформенность. Unity позволяет создавать игры для многих платформ, включая iOS, Android, Windows, Mac, Xbox, PlayStation и многие другие.
    \item Мощный редактор: Unity имеет интуитивно понятный и легко настраиваемый редактор, который облегчает создание игр различных жанров и стилей.
    \item Широкие возможности для разработчиков: Unity поддерживает различные языки программирования, включая C\#, JavaScript, Rust и других, что открывает широкие возможности для разработчиков с разным уровнем опыта.
    \item Обширная библиотека ресурсов: Unity имеет большое сообщество разработчиков, которые делятся своими знаниями, ресурсами и инструментами, доступ к которым значительно упрощает процесс создания игр.
    \item Поддержка различных плагинов и активное развитие: Unity активно развивается, добавляя новые возможности, поддерживая плагины и интегрируя новые технологии, что позволяет разработчикам следить за трендами в индустрии и создавать современные игры.
    \item Доступность. Unity доступен для скачивания и использования бесплатно. Это позволяет начинающим разработчикам с нуля создавать игры.
\end{enumerate}

\textbf{Недостатки:}

\begin{enumerate}
    \item Закрытость кода. Невозможность получения исходных кодов движка даже по лицензии
    \item Не всегда оптимальная производительность: В редких случаях Unity может иметь проблемы с оптимизацией производительности игры, что может привести к низкому FPS или другим проблемам при запуске игры.
    \item Ограниченные возможности для разработки больших проектов: Некоторые разработчики могут столкнуться с ограничениями в создании и управлении большими 
    проектами в Unity из-за его архитектуры и ограниченных инструментов.
\end{enumerate}

\section{\label{sec:ch01/sec02}Кроссплатформенность}

Кроссплатформенность— способность программного обеспечения работать с несколькими аппаратными платформами или операционными системами. Обеспечивается благодаря использованию высокоуровневых
языков программирования, сред разработки и выполнения, поддерживающих
условную компиляцию, компоновку и выполнение кода для различных платформ. Типичным примером является программное обеспечение, предназначенное для работы в операционных системах Linux и Windows одновременно.
Unity поддерживает разработку кроссплатформенных приложений
благодаря своей гибкости и множеству доступных библиотек и инструментов

\section{\label{sec:ch01/sec02}Основные характеристики языка программирования C\#}
C\# – это объектно-ориентированный язык программирования, разработанный компанией Microsoft. Он используется для создания различных типов приложений, включая настольные приложения, веб-приложения, игры, мобильные приложения и другие программы. C\# является частью платформы .NET и широко применяется разработчиками по всему миру. Язык C\# обладает современным синтаксисом, поддерживает множество возможностей, включая управление памятью, многопоточность, обработку исключений и другие современные технологии. Он является одним из популярных языков программирования и широко используется в индустрии разработки ПО.

C\# является бесплатным и открытым исходным кодом, что делает
его доступным для всех.

Также C\# можно использовать для работы с кроссплатворменными приложениями. Это означает, что приложения, разработанные на C\#, могут быть запущены на различных операционных системах без необходимости внесения изменений в исходный код.

C\# имеет несколько недостатков, включая зависимость от экосистемы Microsoft, что ограничивает его использование для тех, кто предпочитает открытые или кросс-платформенные решения. Он также сложен для освоения из-за множества продвинутых возможностей, что усложняет обучение и требует постоянного обновления знаний. Производительность C\# может уступать языкам, компилируемым непосредственно в машинный код, таким как C++. Несмотря на улучшения в кросс-платформенности, существуют проблемы с несовместимостью библиотек и инструментов. Стоимость лицензирования некоторых инструментов и сред разработки также может быть значительной, особенно для небольших компаний и индивидуальных разработчиков.

\section{\label{sec:ch01/sec02}Обоснование и выбор технических средств}

Для разработки игры был использован компьютер со следующими характеристиками:
\begin{itemize}
    \item Процессор: Intel i5 11400F
    \item Видеокарта: MSI GeForce RTX 3060
    \item Оперативная память: 16gb DDR4
    \item HDD: TOSHIBA HDWV110 1 TB
    \item HDD: Hitachi hua722010cla330 1TB
    \item SSD: Smartbuy 120 GB
    \item Операционная система: Windows 10 Pro
    \end{itemize}

На основании предоставленных характеристик компьютера, можно сказать,
что этот компьютер обладает достаточной мощностью для разработки программы в Unity. Характеристики компьютера, такие как процессор Intel i5 11400F и видеокарта  GeForce RTX 3060, являются достаточно современными и обеспечивают хорошую производительность при работе с Unity. 16 GB оперативной памяти и 120 GB SSD также являются достаточными для эффективной работы с проектами в Unity, позволяя быстро компилировать, запускать и изменять проекты. Таким образом, данный компьютер с указанными характеристиками является приемлемым для разработки программы в Unity.